\begin{abstract}
Un actividad muy usual en la vida de las personas es el acto de comprar alimentos o productos de higiene y limpieza. Cualquiera de ellos conlleva la tarea no solo de la compra en si misma sino la elección del producto dentro de una gran variedad y a su vez si escalamos un nivel mas, la selección del establecimiento y la locación del mismo donde se llevará acabo la operación de compra.\\

La Ciudad de Buenos Aires, como centro neurálgico del país presenta una gran oferta de establecimiento para la venta y su concentración, por la poca superficie de la ciudad, es la mayor del país. Si los precios son homogéneos entre los distintos barrios o si la distribución de los locales o presencia de las cadenas en las distintas zonas también lo es, no esta claro de antemano para el comprador que solo busca satisfacer una necesidad cotidiana. Sin embargo, sera el foco de este trabajo abordar la problemática desde dos ejes por separado.\\

El primero hará foto en una análisis sobre información para un periodo de tiempo determinado sobre que se esta vendiendo, a que precio y en que lugares de la Ciudad y por medio de que banderas o empresas de comercialización, el segundo buscara entender por medio de modelos de análisis de datos, como la variables quizás mas importante al momento de realizar una compra esta relacionada con el resto de las variables o características del producto, claro estamos hablando del precio.\\

El precio que según se entiende es en muchos casos el motor para dirimir una decisión o para definir una compra, sera en el transcurso del presente trabajo variable sin lugar a dudas de relevancia y a la que se le presentara particular atención.\\


\emph{\textbf{Keywords: precio, barrio, regresión, supermercado, residuo, boxplot, error}}
\end{abstract}

