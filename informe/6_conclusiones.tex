
\section{Conclusiones}

\begin{itemize}
  \item En todo trabajo de análisis de datos es fundamental tomar contexto no solo con lo que expresan los datos sino también con el momento y la forma que fueron tomados. En el caso del presente trabajo si bien tenemos datos de algunos meses atrás, los resultados y conclusiones que podemos obtener a partir de ellos sigue siendo validas ya que es un conjunto de datos que se tomo en un mismo lugar, con una misma unidad de medición y de la misma forma, lo que asegura que por mas que los mismo envejezcan se pueda aplicar una corrección por el índice de precios o inflación y los resultados pueden seguir resultando significativos. Que los datos hayan sido tomados en una determinada zona del país acotada, también ayuda a dicho análisis.
  \item El enriquecimiento del dataset original por medio de fuentes externas genero un compendio de información mucho mas solida, lo que permitió en muchos casos reemplazar las variables originales por variables categóricas que facilitaban el estudio, por nombran un caso el hecho de trabajar con el barrio en vez de con la dirección o la latitud y longitud que son mas difíciles de comprar resulto en una gran ayuda no solo para el análisis sino también para mostrar los resultados.
  \item Surgió del análisis exploratorio que la evolución de los productos estudiados tuvieron un comportamiento en alza durante el periodo en cuestión al igual que el valor de la cotización promedio compra y venta del Dolar Estadounidense y no así como fluctuó la inflación para el mismo periodo, la cual tuvo picos altos y bajos.
  \item En cuanto a la distribución de los puntos de venta es claro por lo expuesto que los barrios que componente el centro y norte de la Ciudad que concentran los valores de metro cuadrado mas costosos son los que tienen mas puntos de venta.
  \item Es interesante el resultado que expresa que los hipermercados tienen en promedio precios mas altos que los supermercados.
  \item Es errado intentar explicar la variable precio solo por el lugar donde se vende un producto, por la razón social de la empresa que lo comercializa o por el tipo de sucursal, el fenómeno de la conformación de los precios es mucho mas complejo y requiere un análisis en conjunto de todas las variables disponibles inclusive las externas como ser el valor de la inflación mes a mes y el valor dolar.
  \item Sobre el trabajo de los modelos, haber tenido un primer modelo de linea base con una regresión muy sencilla y fácil de interpretar funciono muy bien para obtener rápidamente un modelo que permitiera generar toda la batería de código para la predicción, evaluación del modelo, separación del set de datos, etc.
  \item El mejor modelo fue el de regresión lineal compuesta, aunque los resultados
  de las dos regresiones están muy cercanos. Algo que resulta interesante es que ninguna de los dos regresiones logro quitar covariables, lo cual indica que las elegidas para los modelos tenían el suficiente peso como para sobrevivir a los mismos.\\
  Como era de esperarse cualquiera de estos modelos fue muy superior a la regresión lineal simple, ya que como antes se menciono el precio de los productos no guarda relación únicamente con un predictor.

  
 

\end{itemize}


\section{Criticas y propuestas}

\begin{itemize}
  \item Si bien la masa de datos iniciales parecía suficiente, con aproximadamente 1.5 millones de registros, dichos registros son sobre 1000 productos de venta, si bien esto puede tener cierto sentido ya que son los productos presentes en la mayoría de los puntos de venta, seria interesante contar con la totalidad de los productos comercializados, ya que restringir el estudio a 1000 productos puede producir un sesgo en el análisis. Por ejemplo, se podría estar dejando afuera productos que subieron abruptamente su precio en el periodo estudiado modificando el comportamiento en un barrio o una cadena.
  \item El hecho de haber cierta cantidad de mediciones en un intervalo de tiempo acotado limita el trabajo, lo ideal hubiese sido tener mayor intervalo de tiempo como ser uno o dos años, para así evitar la estacionalidad de algunos momentos (por ejemplo en los datos gran parte de los datos proviene de Diciembre, Enero y Febrero que son meses atípicos en la Ciudad). Y por otro lado, mediciones mas próximas, por ejemplo todas las semanas para enriquecer aun mas el análisis.
  \item Dentro de los puntos de venta, están diferenciados los Supermercados de los Hipermercados, dejando afuera cadenas mas pequeñas o los conocidos negocios de cercanía que en volumen son la mayor cantidad de la ciudad y los que reúnen un gran caudal de visitar y ventas.
  \item Para ampliar el estudio, si bien la Ciudad de Buenos Aires es un buen nicho para estudio, seria interesante contar con otra zona del país para contrarrestar estos resultados, por poner un ejemplo la cadena de venta que mas aumento los precios en la Ciudad tuvo el mismo comportamiento en otra ciudad del país?
  \item Como enriquecimiento hubiera sido interesante contar con los datos de las ventas, no solo saber cuales son los productos que estaban en góndola sino su volumen de ventas, para estudiar si un aumento en los precios generales modifica la compra de un producto o grupo de productos, o incluir el volumen de venta como predictor a los modelos.
  \item Cuando se empezó a trabajar con los modelos se hizo un pre análisis haciendo subset selection e incluyendo nuevas variables, la finalidad de esto es identificar entre todas las variables o predictores posibles las que aporten mas información o estén generando una mayor influencia sobre la variable a predecir.\\
  Para mejorar la potencia de los modelos encontrados hasta el momento, o para desafiar los resultados se podría volver a intentar un subset selection u otras técnicas como ser backward o forward selection a modo de mejorar aun mas la selección de los predictores.
  \item Para la regresión lineal simple, si bien informalmente se probaron algunas otras opciones a la variable banderaDescripcion, se podría probar tomar como baseline la variable que obtenga el mejor modelo en soledad. La idea de hacerlo con banderaDescripcion, era que permitía un analisis mas detallado de las variables individuales dummy, ya que no generaba tantas como por ejemplo la variable marca.
  \item En las regularización se estudio tanto Lasso como Ridge, pero se podría haber hecho un estudio de Elastic-Net que es un método hermano a ellos dos para ver si la performance de los resultados mejoraba.

\end{itemize}