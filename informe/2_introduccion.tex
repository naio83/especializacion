Contar con la información sobre cómo se distribuyen los precios de los productos en la Ciudad de Buenos Aires en los distintos locales de venta, nos puede ayudar en diferentes niveles. Desde el acto más simple de ir a comprar en una cadera determinada, porque su promedio de precios es mucho mejor que otra, elegir un barrio determinado para vivir o trabajar dependiendo de la oferta de puntos de venta que tenga, hasta decidir donde poner un emprendimiento basado en información que se pueda obtener sobre qué productos y en qué cantidad se pueden obtener en una zona determinada.\\
Sabemos que la Ciudad de Buenos Aires, esta dividida en Comunas y mas granularmente en Barrios, los mismos si bien con heterogeneidades mantienen un promedio de vida dentro de los mismos. Por otro lado, existen diversas cadenas de venta o banderas que desde hace años se encargan de la comercialización de productos para el hogar y consumo, todas ellas conforman la red de ofertas de la Ciudad. \\


\section{Objetivos}
Lo que se buscará con este trabajo, en concordancia con lo antes mencionado, es en primer lugar conducir un análisis exploratorio que tiene como ejes poder entender los siguientes puntos, con los análisis secundarios pertinentes, a modo de poder ampliar el contexto de los ejes de la investigación:

\begin{itemize}
	\item Barrios Porteños
    \begin{itemize}
    	\item Variación en cantidad y razón social de los establecimientos de venta según su ubicación.
        \item Variación de precios promedio para los productos o un subconjunto representativo de los mismos.
    \end{itemize}
    \item Empresa o razón social
    \begin{itemize}
    	\item Comportamiento de los productos o subconjunto de los mismos según tipo de comercio (Supermercado o Hipermercado). 
    \end{itemize}
    \item Precios
    \begin{itemize}
    	\item  Investigar cómo es la evolución de los precios para el intervalo de tiempo, pudiendo compara dicha evolución con otras variables externas.
    	\item Relación entre la evolución de los precios y variables externas.
    \end{itemize}
\end{itemize}


Una vez concluido el mismo el trabajo se focalizará en estudiar varios métodos para explicar la variable precio sobre un banco de pruebas y en base a una métrica común poder contrarrestarlos para entender:


\begin{itemize}
    	\item Si las variables originales son o no adecuadas para explicar el precio de un producto.
    	\item La elección de los métodos en el contexto de los experimentos.
        \item Cúal de ellos es quien mejor explica dicha variable y tiene mejor potencia de predicción sobre la misma.
\end{itemize}



En los sucesivos capítulos se presentará un análisis exploratorio de los datos obtenidos, los resultados de los diferente análisis para terminar con las conclusiones finales.






