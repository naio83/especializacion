

Para el análisis antes mencionado se realizará un estudio del comportamiento de los precios de mil productos vendidos en 175 sucursales de todos los barrios de la Ciudad de Buenos Aires, no se pretenden utilizar solamente los datos de venta de los productos sino enriquecer esta colección de información con datos externos obtenidos de fuentes publicas y privadas a modo de hacer mas rico el análisis. La idea de esta incorporación de datos no es simplemente engordar los datos con información de contexto, sino poder cruzar los datos originales con distintos fenómenos presentes en el periodo de tiempo estudiado, como ser la inflación y el precio del dolar norteamericano y también poder tomar datos de distribución de barrios y centros de venta en la ciudad.

Algunos análisis secundarios que se espera poder llevar acabo para poder ampliar los tres ejes de la investigación principal antes citada y que ayudaran a entender podrán ser:


\begin{itemize}
	\item Barrios Porteños
    \begin{itemize}
    	\item La cantidad y razón social de los establecimientos de venta varia según los barrios ?
        \item Existe variación de precios promedio o para algún subconjunto de productos elegidos como representativos ?
    \end{itemize}
    \item Empresa o razón social
    \begin{itemize}
    	\item  Existe variación de precios promedio o para algún subconjunto de productos elegidos como representativos dependiendo del tipo de comercio (Supermercado o Hipermercado)?
    	\item
    \end{itemize}
    \item Precios
    \begin{itemize}
    	\item  Investigar como es la evolución de los precios para el intervalo de tiempo, pudiendo compara dicha evolución con otras variables externas.
    	\item Existe relación entre la evolución de los precios y estas variable externas?
    	\item De que depende el precio? Se puede encontrar una o un grupo de variables que expliquen la formación del precio?
    \end{itemize}
\end{itemize}


En los sucesivos capítulos se presentara un análisis previo de los datos obtenidos, los resultados obtenidos de los diferente análisis para terminar con las conclusiones finales.

